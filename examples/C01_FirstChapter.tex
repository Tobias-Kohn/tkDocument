\documentclass[article, jython]{tkDocument}
%% ����

\title{Turtle-graphics}
\author{Tobias~Kohn}

\begin{document}

\maketitle


  While walking around on your screen, the \emph{turtle} leaves a visible trace. By telling the
  turtle how to move--such as ``turn left'' or ``go forward five steps'', you determine the 
  figure it is drawing.
  
  \emph{This is just a very small sample of a larger script, meant to display how to use the
  \texttt{tkDocument}-template.}
  


%%%%%%%%%%%%%%%%%%%%%%%%%%%%%%%%%%%%%%%%%%%%%%%%%%%%%%%%%%%%%%%%%%%%%%%%%%%%%%%%%%%%%%%%%%%%%%%%%%
\section{Moving the Turtle}

\begin{Lernziele}

  \item to write simple programs, and draw graphics using turtle graphics.
  \item to move the turtle using the commands �left�, �right�, and �forward�.

\end{Lernziele}
  \index{left}\index{right}\index{forward}\index{turtle}


  \paragraph{Introduction}
  Your first programs will be a sequence of instructions for the turtle. Once you have completed
  writing your program, the turtle will follow your program line by line and obey each instruction---
  as long as it understands what you want it to do.

  However, since the turtle is really not that clever, you must be very careful with your
  instructions. Even small typing errors will cause the turtle to be lost, and it will immediately
  stop executing your program.


  \paragraph{The Program}
  This is your first program. Enter the commands into the editor, make sure you get all the spaces
  and case right, and then execute the program by clicking the green ``play''-button. The turtle
  should then draw a right-angled triangle.
  \index{gturtle}\index{import}\index{makeTurtle}
  \begin{pythonln}
    from gturtle import *
    
    makeTurtle()
    
    forward(120 * 1.414216)
    left(135)
    forward(120)
    left(90)
    forward(120)
  \end{pythonln}
  In order to use the turtle, you must first load the file (called a ``module'' in Python) containing 
  the turtle and its commands. This file is called ``gturtle''. Once you have loaded the module (using
  �import� as in the first line above), use �makeTurtle()� to open a window for the turtle to draw in.
  \index{module}

  
\clearpage
  
  \begin{wichtig}
    At the beginning of each turtle-program, you need to load the turtle-module, and open a new window:
    \begin{pythonnb}
      from gturtle import *
      makeTurtle()
    \end{pythonnb}
    Afterwards, you are free to combine as many instructions to the turtle as you like. Each instruction
    has to be written on a line of its own.
    Some of the instructions the turtle will always understand are as follows:
    \index{speed}\index{dot}\index{back}
    
    \begin{tabular}{@{}ll@{}}
    \hline
      �forward(s)�  & Move $s$ pixels forward. \\
      �back(s)�     & Move $s$ pixels back. \\
      �left(w)�     & Turn by $w$ degrees left. \\
      �right(w)�    & Turn by $w$ degrees right. \\
      �dot(d)�      & Paint a dot with a diameter of $d$ pixels. \\
      �speed(-1)�   & Make the turtle as fast as possible. \\
    \hline
    \end{tabular}
  \end{wichtig}


\begin{Aufgaben}

\item
  Have the turtle draw a square or a small house.

\end{Aufgaben}

\end{document}